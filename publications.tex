%% Sometimes when a section can't be nicely modelled with the
% \entry[]... mechanism; hack our own
\makerubrichead{학술 발표 및 논문 (Research Publications)}
\hfill

%% Assuming you've already given \addbibresource{own-bib.bib} in the
% main doc. Right? Right???
\nocite{*}

%% If you just want everything in one list
%\printbibliography[heading={none}]

\printbibliography[heading={none},title={Published},type=article]

\printbibliography[heading={none},title={Proceedings}, type=inproceedings]

%\printbibliography[heading={subbibliography},title={In Review},type=misc]

%\printbibliography[heading={subbibliography},title={Books and Chapters},filter={booksandchapters}]

%\printbibliography[heading=subbibliography]

\makerubrichead{특허 (Patents)}
\hfill

\printbibliography[resetnumbers, heading={none},title={Patents}, type=misc]
